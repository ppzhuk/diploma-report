%%%%%%%%%%%%%%%%%%%%%%%%%%%%%%%%%%%%%%%%%%%%%%%%%%%%%%%%%%%%%%%%%%%%%%%%%%%%%%%%
\chapter{Оценка качества}
\label{chap:quality}
%%%%%%%%%%%%%%%%%%%%%%%%%%%%%%%%%%%%%%%%%%%%%%%%%%%%%%%%%%%%%%%%%%%%%%%%%%%%%%%%

Для оценки качества алгоритма использовались следующие метрики: перплексия, количество ВОП, косинус между вопросом и темой, косинус между ответом и темой. Таблица \ref{quality} показывает влияние этапов предобработки и параметров алгоритма на значения метрик.

Видно, что применение эвристик предобработки и фильтров (за исключением фильтра тем) положительно влияет на косинус вопросов и ответов. Изменение параметров в большую сторону позволяет находить меньшее количество более качественных, с точки зрения косинусного расстояние, ВОП.

Экспертная оценка проводилась только для оптимального набора параметров. В качестве эксперта выступил разработчик YouTrack, которому было предложено 20 ВОП с наибольшим значением гармонического среднего. Эксперту необходимо было оценить, какие из предложенных ВОП подходят для публикации в ЧЗВ. Результаты представлены в таблице \ref{qualityExpert}. Доля подходящих для публикации ВОП составила 50\%, для \cite{original} данный показатель составил 37\%. 

Да момент написания статьи работа еще не была завершена, планируется дополнительная оптимизация с целью повышения доли подходящих для публикации ВОП и получение дополнительных экспертных оценок.
\begin{landscape}
\begin{table*}[t]
  \caption{Влияние эвристик и параметров на медианные значения метрик}
  \label{quality}
  \centering
  \begin{tabular}{|c|c|c||c|c|c|c|}
     \hline
      \parbox[t]{3cm}{\textbf{Исследуемый параметр}} &%
     \parbox[t]{2cm}{\textbf{Старое значение}} &%
     \parbox[t]{1.7cm}{\textbf{Новое значение}} &%
     \textbf{Перплексия} &%
     \parbox[t]{2cm}{\textbf{Количество ВОП}} &%
     \textbf{$\cos(Q,T)$} &%
     \textbf{$\cos(A,T)$} \\
	 \hline
     \parbox[t]{2.7cm}{Оптимальные параметры} &%
     - &%
     - &%
	 1864 &%
     357 &%
 	 0.396 &%
     0.413 \\
  	 \hline
     \parbox[t]{3cm}{Эвристики отображения(\ref{sec3}-\ref{sec3-a})} &%
     вкл &%
     выкл &%
	 1971 &%
     361 &%
 	 0.371 &%
     0.395 \\
	 \hline
	 \parbox[t]{3cm}{Эвристики тематич. моделир.(\ref{sec3}-\ref{sec3-b})} &%
     вкл &%
     выкл &%
	 2016 &%
     384 &%
 	 0.369 &%
     0.391 \\
	 \hline
\parbox[t]{3cm}{Фильтр обращений(\ref{sec3}-\ref{sec3-c})} &%
     вкл &%
     выкл &%
	 1924 &%
     403 &%
 	 0.370 &%
     0.388 \\
	 \hline
\parbox[t]{3cm}{Фильтр тем(\ref{sec5}-\ref{sec5-a})} &%
     вкл &%
     выкл &%
	 1791 &%
     472 &%
 	 0.393 &%
     0.416 \\
     \hline
\parbox[t]{2.7cm}{Порог выбора темы LDA(\ref{sec4}-\ref{sec4LDA})} &%
     0.25 &%
     0.0 &%
	 1853 &%
     376 &%
 	 0.366 &%
     0.404 \\
     \hline
\parbox[t]{2.7cm}{Порог выбора темы LDA(\ref{sec4}-\ref{sec4LDA})} &%
     0.25 &%
     0.4 &%
	 1871 &%
     302 &%
 	 0.399 &%
     0.421 \\
     \hline
\parbox[t]{3cm}{Минимальное значение косинуса(\ref{sec5}-\ref{sec5-b})} &%
     0.15 &%
     0.0 &%
	 1860 &%
     394 &%
 	 0.337 &%
     0.359 \\
     \hline
\parbox[t]{3cm}{Минимальное значение косинуса(\ref{sec5}-\ref{sec5-b})} &%
     0.15 &%
     0.3 &%
	 1864 &%
     288 &%
 	 0.387 &%
     0.419 \\
     \hline         
\parbox[t]{3cm}{Минимальная доля ВОП(\ref{sec5}-\ref{sec5-c})} &%
     0.1 &%
     0.0 &%
	 1869 &%
     376 &%
 	 0.384 &%
     0.407 \\
     \hline
\parbox[t]{3cm}{Минимальная доля ВОП(\ref{sec5}-\ref{sec5-c})} &%
     0.1 &%
     0.2 &%
	 1850 &%
     324 &%
 	 0.391 &%
     0.417 \\
     \hline     
  \end{tabular}

\end{table*}
\end{landscape}

\begin{table}[!ht]
\caption{Экспертная оценка}
\label{qualityExpert}
\centering
\begin{tabular}{|c|c|c|}
\hline
Категория ВОП & Количество & Доля, \% \\
\hline
\parbox[t]{4cm}{\textbf{Общее количество}} & 20 & 100 \\

	 \hline
\parbox[t]{4cm}{Подходит для публикации без редактирования} & 6 & 30 \\

	 \hline
\parbox[t]{4cm}{Подходит для публикации с редактированием вопроса или ответа} & 4 & 20\\

	 \hline
\parbox[t]{4cm}{Не подходит для публикации. Некорректный вопрос или ответ} & 10 & 50\\
\hline
\end{tabular}
\end{table}
