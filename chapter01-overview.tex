%%%%%%%%%%%%%%%%%%%%%%%%%%%%%%%%%%%%%%%%%%%%%%%%%%%%%%%%%%%%%%%%%%%%%%%%%%%%%%%%
\chapter{Анализ методов извлечения часто задаваемых вопросов}
\label{chap:overview}
%%%%%%%%%%%%%%%%%%%%%%%%%%%%%%%%%%%%%%%%%%%%%%%%%%%%%%%%%%%%%%%%%%%%%%%%%%%%%%%%

О чем глава?

%%%%%%%%%%%%%%%%%%%%%%%%%%%%%%%%%%%%%%%%%%%%%%%%%%%%%%%%%%%%%%%%%%%%%%%%%%%%%%%%
\section{Существующие подходы к задаче извлечения ЧЗВ}
\label{sec:researches}
%%%%%%%%%%%%%%%%%%%%%%%%%%%%%%%%%%%%%%%%%%%%%%%%%%%%%%%%%%%%%%%%%%%%%%%%%%%%%%%%

Предварительным этапом данной работы стало изучение существующих методик извлечения вопросно-ответных пар. Были изучены различные научные статьи за период 2004--2016 гг., выявлены следующие методы:

\begin{itemize*}
\item Методы, основанные на машинном обучении~---~\cite{ML1},~\cite{ML2};
\item Методы, основанные на классификации~---~\cite{classif1},~\cite{engine};
\item Методы на основе тематического моделирования~---~\cite{LDA1},~\cite{LDA2},~\cite{original}.
\end{itemize*}

Методы, применяющие тематическое моделирование, являются наиболее предпочтительными, поскольку построение и использование тематчиеской модели позволяет находить ВОП, у которых в вопросе и ответе используется схожая терминология. Это позволяет находить более качественные ВОП~\cite{LDA2} по сравнению со способами, использующими машинное обучение или классификацию.

Работа~\cite{original} предлагает решать задачу извлечения ВОП в три шага: предобработка данных, тематическое моделирование, поиск вопросно-ответных пар. От других работ, использующих тематическое моделирование, эту работу отличает наличие дополнительных шагов обработки данных, специфичных для ИТ-дискуссий. При решении поставленной задачи, именно работа~\cite{original} применялась в качестве основной. Данная работа также использует модель скрытого размещения Дирихле (LDA), котороя более подробно рассмотрена далее в этом разделе.

Работы~\cite{LDA1} и~\cite{LDA2} также используют LDA для вопросно-ответных систем. Работа~\cite{LDA1}, однако, предлагает проводить тематическое моделирование в рамках одного обращения, что может дать менее качественный результат, поскольку LDA показывает лучшие результаты на больших объемах данных. В \cite{LDA2} LDA используется для определения темы вновь поступивших вопросов, при этом для них не определяется ответ. В статьях~\cite{TMuse} и~\cite{2016lda} рассмотрены модификации LDA, призванные улучшить качество тематического моделирования.

Работа \cite{so} предлагает способ для нахождения лучшего ответа  на вопрос среди уже предоставленных на примере размеченных данных со Stack Overflow. В текущей работе не используются размеченные данные, что позволяет получить более универсальное решение. В статье \cite{engine} представлен другой подход поиска ответов, связанный с использованием поисковой системы. Сначала комментарии разделяются на 6 классов: вопрос, уточнение, ответ, отзыв на ответ, мусор. Затем используется специально настроенная поисковая система для поиска только по ответам. Основное отличие от текущей работы заключается в способе определения релевантных ответов.

%%%%%%%%%%%%%%%%%%%%%%%%%%%%%%%%%%%%%%%%%%%%%%%%%%%%%%%%%%%%%%%%%%%%%%%%%%%%%%%%
\section{Тематическое моделировлания}
\label{sec:overview_tm}
%%%%%%%%%%%%%%%%%%%%%%%%%%%%%%%%%%%%%%%%%%%%%%%%%%%%%%%%%%%%%%%%%%%%%%%%%%%%%%%%

Тематическое моделирование~--- способ построения модели коллекции текстовых документов, которая определяет, к каким темам относится каждый из документов~\cite{TM}. 

Впервые задача тематического моделирования возникла в 1958 году, когда Герхард Лисовски и Леонард Рост завершили работу по составлению каталога религиозных текстов на иврите, которые должны были помочь учёным определить значения давно утраченых терминов. Затем они собрали вместе все возможные контексты, в которых появлялся каждый из терминов. Следующей задачей было научиться игнорировать несущественные различия в формах слов и выделять те различия, которые влияют на семантику. Замыслом авторов было дать возможность исследователям языка проанализировать различные отрывки и понять семантику каждого термина в его контексте. 

Проблемы такого рода возникают и сегодня при автоматическом анализе текстов. Одна и та же концепция может выражаться любым количеством различных терминов (синонимия), тогда как один термин часто имеет разные смыслы в различных контекстах (полисемия). Таким образом, необходимы способы различать варианты представления одной концепции и определять конкретный смысл многозначных терминов. Теоретически обоснованным и активно развивающимся направлением в анализе текстов на естественном языке, призванным решать перечисленные задачи, является тематическое моделирование коллекций текстовых документов.   

Построение тематической модели может рассматриваться как задача одновременной кластеризации документов и слов по одному и тому же множеству кластеров, называемых темами. В терминах кластерного анализа тема — это результат би-кластеризации, то есть одновременной кластеризации и слов, и документов по их семантической близости. Обычно выполняется нечёткая кластеризация, то есть документ может принадлежать нескольким темам в различной степени. Таким образом, сжатое семантическое описание слова или документа представляет собой вероятностное распределение на множестве тем. Процесс нахождения этих распределений называется тематическим моделированием. 

Тематическая модель (англ. topic model) коллекции текстовых документов определяет, к каким темам относится каждый документ и какие слова~(термины) образуют каждую тему~\cite{ML_PTM}. 

Переход из пространства терминов в пространство найденных тематик помогает эффективнее решать такие задачи, как тематический поиск, классификация и аннотация коллекций документов и новостных потоков.

Тематическое моделирование как вид статистических моделей для нахождения скрытых тем встреченных в коллекции документов, нашло свое применение в таких областях как машинное обучение и обработка естественного языка. Исследователи используют различные тематические модели для анализа текстов, текстовых архивов документов, для анализа изменения тем в наборах документов. В случае, когда документ относится к определенной теме, в документах посвященных одной теме можно встретить некоторые слова чаще других. 

Например: <<собака>> и <<кость>> встречаются чаще в документах про собак, «кошки» и «молоко» будут встречаться в документах о коошках, предлоги «и» и «в» будут встречаться в обеих тематиках. Обычно документ касается нескольких тем в разных пропорциях, таким образом, документ в котором 10\% темы составляют кошки, а 90\% темы про собак, можно предположить, что слов про собак в 9 раз больше. Тематическое моделирование отражает эту интуицию в математическую структуру, которая позволяет на основании изучения коллекции документов и исследования частотных характеристик слов в каждом документе, сделать вывод, что каждый документ это некоторый баланс тем.

Как правило, количество тем, встречающихся в документах, меньше количества различных слов во всем наборе. Поэтому скрытые переменные (темы) позволяют представить документ в виде вектора в пространстве скрытых (латентных) тем вместо представления в пространстве слов. В результате документ имеет меньшее число компонент, что позволяет быстрее и эффективнее его обрабатывать. Таким образом, тематическое моделирование также может использоваться в таком классе задач, как уменьшение размерности данных. Кроме того, найденные темы могут использоваться для семантического анализа текстов. 

%%%%%%%%%%%%%%%%%%%%%%%%%%%%%%%%%%%%%%%%%%%%%%%%%%%%%%%%%%%%%%%%%%%%%%%%%%%%%%%%
\section{Методы построения тематической модели}
\label{sec:tm_techniques}
%%%%%%%%%%%%%%%%%%%%%%%%%%%%%%%%%%%%%%%%%%%%%%%%%%%%%%%%%%%%%%%%%%%%%%%%%%%%%%%%

Задача извлечения скрытых тем из коллекции текстовых документов имеет множество применений. Помимо кластеризации и классификации документов, найденные темы могут применяться для определения релевантности документа заданной теме или запросу, определения тематического сходства документа с другими документами и их фрагментами, построения тематических профилей авторов, разбиения документа на тематически однородные фрагменты и т.д. 

Современные способы тематического моделирования находят применение в широком спектр приложений: 

\begin{itemize*}
\item Кластеризация, классификация, ранжирование, аннотирование и суммаризация отчётов, научных публикаций, архивов докуметов и т.д.; 
\item Тематический поиск документов и связанных с ними объектов;
\item Фильтрация спама; 
\item Построение тематических профилей пользователей форумов, блогов и социальных сетей для поиска тематических сообществ и определения наиболее активных их участников; 
\item Анализ новостных потоков и сообщений из социальных сетей для определения актуальных событий реального мира и реакции пользователей на них.
\end{itemize*}

Тематическое моделирование позволяет автоматически систематизировать и обрабатывать электронные архивы такого масштаба, который человек не в силах обработать.

С точки зрения поставленной в данной работе задачи, тематчиеское моделирование используется для построения тематической модели корпуса обращений в службу поддержки. Затем в рамках одного обращения определяется пара комментариев, наиболее близкая к соответсвующей данному обращению теме. Таким образом учитывается сходство терминологии между задаваемым вопросом и полученным ответом на него.

\subsection{Кластеризация и классификация}
\subsection{Латентно-семантическое индексирование}
\subsection{Вероятностный латентно-семантический анализ}
\subsection{Латентное размещение Дирихле}

Лда. Почему именно ЛДА

\subsection{Другие методы}


\section{Оценка качества тематической модели}
\label{sec:tm_techniques}


%%%%%%%%%%%%%%%%%%%%%%%%%%%%%%%%%%%%%%%%%%%%%%%%%%%%%%%%%%%%%%%%%%%%%%%%%%%%%%%%
\section{Вывод}
\label{sec:overview_concl}
%%%%%%%%%%%%%%%%%%%%%%%%%%%%%%%%%%%%%%%%%%%%%%%%%%%%%%%%%%%%%%%%%%%%%%%%%%%%%%%%

\blindtext