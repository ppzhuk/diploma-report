%%%%%%%%%%%%%%%%%%%%%%%%%%%%%%%%%%%%%%%%%%%%%%%%%%%%%%%%%%%%%%%%%%%%%%%%%%%%%%%%
\chapter{Обзор этапов подхода}
\label{chap:overview}
%%%%%%%%%%%%%%%%%%%%%%%%%%%%%%%%%%%%%%%%%%%%%%%%%%%%%%%%%%%%%%%%%%%%%%%%%%%%%%%%
\section{Обзор этапов подхода}
\label{sec:overview}
%%%%%%%%%%%%%%%%%%%%%%%%%%%%%%%%%%%%%%%%%%%%%%%%%%%%%%%%%%%%%%%%%%%%%%%%%%%%%%%%
Этап 1 --- подготовка обращений к отображению в ЧЗВ. Сырые данные содержат много шума: HTML разметка, заголовки электронной почты (<<01 июля 2001 г., 10:10 пользователь ... написал:>>), приветствия, благодарности и так далее. Этот шум заметно влияет на алгоритм. Для повышения качества результатов применяется ряд эвристик предобработки, которые значительно доработаны в сравнение со статьей \cite{original}. При этом также фильтруются обращения, которые заведомо не могут содержать вопроса или ответа. 

Этап 2 --- определение кластеров связанных обращений (в дальнейшем --- тем) используется скрытое размещение Дирихле. LDA описывает каждую тему с помощью мешка слов (наиболее характерных терминов) и для каждого обращения определяет вероятностное распределение по темам. Затем каждому обращению сопоставляется тема с наибольшей вероятностью. После чего темы проходят через фильтр, с целью удаления незначимых с точки зрения ЧЗВ тем.

Этап 3 --- формирование ВОП. Для каждого комментария в рамках обращения считается метрика близости между текстом комментария и соответствующей темой. На основе этой метрики определяются хорошо сформулированные вопросы и релевантные ответы на них.

Стоит отметить, что хотя темематическое моделирование и позволяет определить схожие по терминологии вопросы, которые фактичечски являются часто задаваемыми, алгоритм не ограничивается только этим и позволяет находить редкие ВОП, если они хорошо сформулированы и имеют корректный ответ. При этом большее внимание в работе уделялось поиску качественных ВОП (точность), чем поиску всех возможных ВОП (полнота). Основная мотивация такого решения заключается в желании сократить до минимума ручную часть алгоритма --- валидацию и редактирование ВОП.
%%%%%%%%%%%%%%%%%%%%%%%%%%%%%%%%%%%%%%%%%%%%%%%%%%%%%%%%%%%%%%%%%%%%%%%%%%%%%%%%
\section{Связанные работы}
\label{sec:researches}
%%%%%%%%%%%%%%%%%%%%%%%%%%%%%%%%%%%%%%%%%%%%%%%%%%%%%%%%%%%%%%%%%%%%%%%%%%%%%%%%
Работы \cite{LDA1} и \cite{LDA2} также используют LDA для вопросно-ответных систем. Работа \cite{LDA1}, однако, предлагает проводить тематическое моделирование в рамках одного обращения, что может дать менее качественный результат, поскольку LDA показывает лучшие результаты на больших объемах данных. В \cite{LDA2} LDA используется для определения темы вновь поступивших вопросов, при этом для них не определяется ответ.

Работа \cite{so} предлагает способ для нахождения лучшего ответа  на вопрос среди уже предоставленных на примере размеченных данных со Stack Overflow. В текущей работе не используются размеченные данные, что позволяет получить более универсальное решение. В статье \cite{engine} представлен другой подход поиска ответов, связанный с использованием поисковой системы. Сначала коммментарии разделяются на 6 классов: вопрос, уточнение, ответ, отзыв на ответ, мусор. Затем используется специально настроенная поисковая система для поиска только по ответам. Основное отличие от текущей работы заключается в способе определения релевантных ответов.