%%%%%%%%%%%%%%%%%%%%%%%%%%%%%%%%%%%%%%%%%%%%%%%%%%%%%%%%%%%%%%%%%%%%%%%%%%%%%%%%
\chapter{Анализ методов извлечения часто задаваемых вопросов}
\label{chap:overview}
%%%%%%%%%%%%%%%%%%%%%%%%%%%%%%%%%%%%%%%%%%%%%%%%%%%%%%%%%%%%%%%%%%%%%%%%%%%%%%%%

О чем глава?

%%%%%%%%%%%%%%%%%%%%%%%%%%%%%%%%%%%%%%%%%%%%%%%%%%%%%%%%%%%%%%%%%%%%%%%%%%%%%%%%
\section{Обзор существующих подходов к задаче извлечения ЧЗВ}
\label{sec:researches}
%%%%%%%%%%%%%%%%%%%%%%%%%%%%%%%%%%%%%%%%%%%%%%%%%%%%%%%%%%%%%%%%%%%%%%%%%%%%%%%%

Более менее подробный обзор статей по теме + вывод почему наш выбор именно такой

Работы~\cite{LDA1} и~\cite{LDA2} также используют LDA для вопросно-ответных систем. Работа~\cite{LDA1}, однако, предлагает проводить тематическое моделирование в рамках одного обращения, что может дать менее качественный результат, поскольку LDA показывает лучшие результаты на больших объемах данных. В~\cite{LDA2} LDA используется для определения темы вновь поступивших вопросов, при этом для них не определяется ответ.

Работа~\cite{so} предлагает способ для нахождения лучшего ответа  на вопрос среди уже предоставленных на примере размеченных данных со Stack Overflow. В текущей работе не используются размеченные данные, что позволяет получить более универсальное решение. В статье~\cite{engine} представлен другой подход поиска ответов, связанный с использованием поисковой системы. Сначала коммментарии разделяются на 6 классов: вопрос, уточнение, ответ, отзыв на ответ, мусор. Затем используется специально настроенная поисковая система для поиска только по ответам. Основное отличие от текущей работы заключается в способе определения релевантных ответов.

%%%%%%%%%%%%%%%%%%%%%%%%%%%%%%%%%%%%%%%%%%%%%%%%%%%%%%%%%%%%%%%%%%%%%%%%%%%%%%%%
\section{Задача тематического моделировлания}
\label{sec:overview_tm}
%%%%%%%%%%%%%%%%%%%%%%%%%%%%%%%%%%%%%%%%%%%%%%%%%%%%%%%%%%%%%%%%%%%%%%%%%%%%%%%%

Что такое ТМ и почему это основной этап

%%%%%%%%%%%%%%%%%%%%%%%%%%%%%%%%%%%%%%%%%%%%%%%%%%%%%%%%%%%%%%%%%%%%%%%%%%%%%%%%
\section{Обзор методов тематического моделирования}
\label{sec:tm_techniques}
%%%%%%%%%%%%%%%%%%%%%%%%%%%%%%%%%%%%%%%%%%%%%%%%%%%%%%%%%%%%%%%%%%%%%%%%%%%%%%%%

\subsection{метод 1}
\subsection{метод 2}
\subsection{метод 3}

Почему именно ЛДА

%%%%%%%%%%%%%%%%%%%%%%%%%%%%%%%%%%%%%%%%%%%%%%%%%%%%%%%%%%%%%%%%%%%%%%%%%%%%%%%%
\section{Вывод}
\label{sec:overview_concl}
%%%%%%%%%%%%%%%%%%%%%%%%%%%%%%%%%%%%%%%%%%%%%%%%%%%%%%%%%%%%%%%%%%%%%%%%%%%%%%%%

\blindtext