%%%%%%%%%%%%%%%%%%%%%%%%%%%%%%%%%%%%%%%%%%%%%%%%%%%%%%%%%%%%%%%%%%%%%%%%%%%%%%%%
\conclusion
%%%%%%%%%%%%%%%%%%%%%%%%%%%%%%%%%%%%%%%%%%%%%%%%%%%%%%%%%%%%%%%%%%%%%%%%%%%%%%%%
В результате выполнения данной работы была разработана технология извлечения вопросно-ответных пар из обращений в службу поддержки. Предложенный способ учитывает специфику ИТ-дискуссий и, помимо обращений в службу поддержки, может быть также применен к таким источникам как: вопросно-ответные системы, форумы.

Для решения данной задачи был предложен подход с использованием тематического моделирования. Для каждого обращения определяется тема, которая представляет собой мешок слов (секция~\ref{sec:overview}). Вопросно-ответные пары затем определяются на основе косинусного расстояния между отдельно взятым комментарием и темой обращения. Данный подход основан на сходстве используемой терминологии между вопросом и ответом на него.

В работе были рассмотрены существующие на сегодняшний день тематические модели (раздел 1). Проведено их сравнение, на основе которого выбрана модель LDA, как наиболее эффективная. Важными особенностями данной модели является то, что она не требует размеченных входных данных для обучения и позволяет определять тему для нового документа без необходимости повторного обучения.

В разделе 2 описаны анализируемые данные и их источник. Формируются решаемые задачи: 

\begin{enumerate*}
\item Предобработка данных;
\item Кластеризация обращений по темам;
\item Извлечение ВОП;
\item Оценка эффективности разработанного подхода.
\end{enumerate*}

Первые три шага подробно описаны разделе 3. Предобработка данных позволяет избавиться от обращений, заведомо не содержащих ВОП. Из оставшихся обращений удаляются шумы, которые негативно влияют на качество тематической модели и затрудняют определение вопроса или ответа. Применяются специфичные для ИТ-дискуссий эвристики, связанные с удалением машинно-сгенерированного текста. 

LDA используется для кластеризации обращений по темам. Каждому обращению присваивается тема с наибольшей вероятностью, после чего происходит определение ВОП.

В разделе 4 приведена реализация соответствующего предложенному решению алгоритма. Эта реализация используется в разделе 5 для определения качества извлекаемых ВОП и эффективности решения.

Оценка влияния эвристик и параметров на качество извлекаемых ВОП показала, что используемые эвристики влияют на их качество положительно. Повышение значений параметров алгоритма приводит к уменьшению результирующего количества ВОП, при этом повышаются значения выбранных метрик качества.

По результатам экспертной оценки, 78\% из найденных ВОП признано корректными, команда YouTrack заинтересована во внедрении данного решения в рабочий процесс.

К достоинствам описанного в данной работе подхода можно отнести то, что помимо часто задаваемых вопросов могут быть также найдены и редкие ВОП, если они хорошо сформулированы и имеют корректный ответ.

К ограничениям можно отнести: анализируются только обращения на английском языке; в качестве ответа выбирается только один комментарий. 

Если ответ на вопрос (или сам вопрос) содержатся в более чем одном комментарии, то, вероятно, вопрос плохо сформулирован или ответ недостаточно полон. Однако комбинирование нескольких комментариев для составления вопроса или ответа является одной из возможных тем для дальнейших исследований. К другим направлениям для дальнейших исследований относятся:

\begin{itemize*}
\item Увеличение полноты решения;
\item Улучшение качества определения машинно-сгенерированного текста;
\item Обработка мультиязычных данных: 
\begin{itemize*}
\item Моноязычные обращения; 
\item Мультиязычные обращения.
\end{itemize*}
\end{itemize*}
