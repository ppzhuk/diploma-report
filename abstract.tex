
\keywords{%
анализ текста,
обработка естественного языка,
тематическое моделирование,
часто задаваемые вопросы
}

\abstractcontent{
Часто задаваемые вопросы (ЧЗВ) содержат актуальную информацию о программном продукте и позволяют снизить нагрузку на отдел технической поддержки. Формирование ЧЗВ и поддержка их в актуальном состоянии требует существенных затрат от разработчика. 

Описываемый в данной работе способ позволяет в автоматическом режиме выбрать наиболее релевантные для добавления в ЧЗВ вопросно-ответные пары, которые затем передаются эксперту для редактирования перед публикацией. Для этого применяются методы интеллектуального анализа текста и тематического моделирования. 

Данный подход может быть применен и для других источников ИТ-дискуссий, таких как: форумы, вопросно-ответные системы. Практические результаты показывают, что используемый подход позволяет упростить формирование актуальных ЧЗВ.
}

\nomenclature{ЧЗВ}{Часто Задаваемые Вопросы}

\keywordsen{
text mining,
natural language processing,
topic modeling,
frequently asked questions
}

\abstractcontenten{
Frequently asked questions (FAQ) contains answers for typical user problems of the software product and helps to decrease amount of calls to user support department. Creating the FAQ and filling out it with the actual information is pretty time- and resource-consuming for the developer.

This work proposes the technique for automatic extraction of the most relevant for the adding in the FAQ question-answer pairs. Then extracted question-answer pairs should be validated or edited by the expert before publication. The technique is based on the text mining and topic modeling approaches. 

It also could be applied for the other IT-discussions sources such as forums, question-answers systems and so on. Practical results show that this technique can be used to facilitate the creation of the FAQs.
}

\nomenclature{FAQ}{Frequently Asked Questions, часто задаваемые вопросы}