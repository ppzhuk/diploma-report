%%%%%%%%%%%%%%%%%%%%%%%%%%%%%%%%%%%%%%%%%%%%%%%%%%%%%%%%%%%%%%%%%%%%%%%%%%%%%%%%
\intro
%%%%%%%%%%%%%%%%%%%%%%%%%%%%%%%%%%%%%%%%%%%%%%%%%%%%%%%%%%%%%%%%%%%%%%%%%%%%%%%%

Часто задаваемые вопросы (ЧЗВ) --- список вопросов, которые часто возникают по какой-либо теме, и ответы на них, данные экспертами в соответствующей области. Програмное обеспечение может сопровождаться ЧЗВ для помощи пользователям в решении распространенных проблем, например: Linux \footnote{http://tldp.org/FAQ/Linux-FAQ/index.html}, Apache Lucene\footnote{https://wiki.apache.org/lucene-java/LuceneFAQ}, Eclipse SWT\footnote{http://www.eclipse.org/swt/faq.php}.

Основное преимущество ЧЗВ над пользовательской документацией и тематическими форумами --- простота поиска необходимой информации. Однако создание качественных ЧЗВ --- это нетривиальный процесс, требующий либо предугадывания потенциальных вопросов, либо ручного анализа обратной связи от пользователей. Целью проведенного исследования является упрощение данного процесса.

Предлагаемый способ основывается на анализе обращений клиентов в службу поддержки, однако может быть использован и для других источников ИТ-дискуссий: форумов, вопросно-ответных систем. Сначала определяются часто обсуждаемые, повторяющиеся темы, для этого используется тематическое моделирование, а именно --- скрытое размещение Дирихле (Latent Dirichlet allocation, LDA) \cite{LDA}, дополненное шагами пред- и постобработки, специфичными для ИТ-дискусиий. Далее среди обращений, относящихся к одной теме, с помощью косинусного расстояния и дополнительных фильтров проходит поиск пар вопрос-ответ (ВОП).

Извлечение ВОП --- это автоматический процесс, однако перед публикацией в ЧЗВ необходимо провести дополнительный экспертный анализ, поскольку для извлеченных ВОП может потребоваться дополнительная валидиция, переформулирование или редактирование (например, удаление конфеденциальных данных). Таким образом, весь подход является полуавтоматическим.

За основу работы взята статья \cite{original}. Основные отличия данного подхода от предложенного в \cite{original} заключаются в следующем:

\begin{itemize}
\item доработаны и расширены эвристики предобработки;
\item применены дополнительные фильтры до и после LDA;
\item используются входные данные другого типа.
\end{itemize}

В \cite{original} в качестве входных данных используются списки рассылки электронной почты, и эти же данные используются для оценки эффективности алгоритма (каждый список рассылки принят за отдельную тему). В данной работе входные данные имееют более разнообразную природу --- обращения собраны из различных каналов: электронная почта, социальные сети, форма для прямой отправки обращений и так далее. Обращения состоят из комментариев и, в общем случае, представляют собой диалог между клиентом и сотрудником технической поддержки. При этом мы не можем делать предположений о разбиении этих данных по темам.

Для тестирования подхода использовались обращения пользователей в техническую поддержку системы отслеживания ошибок YouTrack\footnote{http://jetbrains.ru/products/youtrack/} за период с декабря 2015 года по сентябрь 2016 года. Было проанализировано 6500 обращений. Стоит отметить, что на момент написания статьи работа еще не была завершена, однако промежуточные результаты показывают, что подход может быть применен для генерации ЧЗВ. Из предложенных экспертам ВОП 50\% было признано корректными в сравнение с 37\%, полученными в работе \cite{original}.
