%%%%%%%%%%%%%%%%%%%%%%%%%%%%%%%%%%%%%%%%%%%%%%%%%%%%%%%%%%%%%%%%%%%%%%%%%%%%%%%%
\intro
%%%%%%%%%%%%%%%%%%%%%%%%%%%%%%%%%%%%%%%%%%%%%%%%%%%%%%%%%%%%%%%%%%%%%%%%%%%%%%%%

Часто задаваемые вопросы (ЧЗВ)~--- список вопросов, которые часто возникают по какой-либо теме, и ответы на них, данные экспертами в соответствующей области. Программное обеспечение может сопровождаться ЧЗВ для помощи пользователям в решении распространенных проблем, например: Linux\footnote{http://tldp.org/FAQ/Linux-FAQ/index.html}, Apache Lucene\footnote{https://wiki.apache.org/lucene-java/LuceneFAQ}, Eclipse SWT\footnote{http://www.eclipse.org/swt/faq.php}.

Основное преимущество ЧЗВ над пользовательской документацией и тематическими форумами~--- простота поиска необходимой информации. Однако создание качественных ЧЗВ~--- это нетривиальный процесс, требующий либо предугадывания потенциальных вопросов, либо ручного анализа обратной связи от пользователей. Целью данной работы является разработка метода извлечения ВОП из обращений в службу поддержки для упрощения задачи формирования ЧЗВ.

Предлагаемый способ, помимо обращений в службу поддержки, может быть использован и для других источников ИТ-дискуссий: форумов, вопросно-ответных систем. Сначала определяются часто обсуждаемые, повторяющиеся темы, для этого используется тематическое моделирование, а именно~--- скрытое размещение Дирихле (Latent Dirichlet allocation, LDA)~\cite{LDA}, дополненное шагами пред- и постобработки, специфичными для ИТ-дискуссий. Далее среди обращений, относящихся к одной теме, с помощью косинусного расстояния и дополнительных фильтров проходит поиск вопросно-ответных пар (ВОП). 

\nomenclature{ВОП}{Вопросно-Ответная Пара}
\nomenclature{LDA}{Latent Dirichlet Allocation, скрытое размещение Дирихле}

Способ извлечения ВОП, предлагаемый в данной работе является автоматическим. Однако перед публикацией в ЧЗВ необходимо провести дополнительный экспертный анализ, поскольку для извлеченных ВОП может потребоваться валидация, переформулирование или редактирование (например, удаление конфиденциальных данных, исправление грамматических ошибок). Таким образом, весь подход является полуавтоматическим. 

Работа состоит из пяти разделов. В разделе 1 рассматриваются существующие подходы к задаче извлечения ЧЗВ, описываются различные тематические модели. Раздел 2 посвящен постановке задачи извлечения ЧЗВ из обращений в службу поддержки и описанию анализируемых данных. В разделе 3 описывается предлагаемый подход к решению поставленной задачи. Представлена общая схема подхода, а также подробно рассмотрен каждый из этапов. В разделе 4 рассматривается разработка реализации алгоритма, соответствующего предложенному способу. Раздел 5 посвящен оценке эффективности полученного решения и качества ВОП. В этом разделе изучается влияние эвристик предобработки и параметров на выбранные метрики качества, а также приводятся результаты экспертной оценки.