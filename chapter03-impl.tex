%%%%%%%%%%%%%%%%%%%%%%%%%%%%%%%%%%%%%%%%%%%%%%%%%%%%%%%%%%%%%%%%%%%%%%%%%%%%%%%%
\chapter{Реализация}
\label{chap:impl}
%%%%%%%%%%%%%%%%%%%%%%%%%%%%%%%%%%%%%%%%%%%%%%%%%%%%%%%%%%%%%%%%%%%%%%%%%%%%%%%%

%%%%%%%%%%%%%%%%%%%%%%%%%%%%%%%%%%%%%%%%%%%%%%%%%%%%%%%%%%%%%%%%%%%%%%%%%%%%%%%%
\section{XML}
%%%%%%%%%%%%%%%%%%%%%%%%%%%%%%%%%%%%%%%%%%%%%%%%%%%%%%%%%%%%%%%%%%%%%%%%%%%%%%%%

\blindtext

You can use all kinds of abbreviations that don't mean anything, but add
a false sense of importance and significance to your work. Some of these
abbreviations are:
%
\begin{itemize*}
\item eXtensible Markup Language~(XML)
\item JavaScript Object Notation~(JSON)
\item Yet Another Markup Language~(YAML)
\end{itemize*}
%
\nomenclature{XML}{eXtensible Markup Language}
\nomenclature{JSON}{JavaScript Object Notation}
\nomenclature{YAML}{Yet Another Markup Language}

\begin{table}
\caption{Решетка замечательности аббревиатур}
\centering
XML < JSON < YAML
\end{table}

\Blindtext

  \begin{lstlisting}[language=Java, label={lst:aspectj_example}, 
  caption={Пример описания аспектов в AspectJ}]
aspect A  {
  pointcut fooPC(): execution(void Test.foo());
  pointcut printPC(): call(void System.out.println(String));
  
  before(): cflow(fooPC()) && printPC() {
    System.out.println("Hello, world!");
  }
}
  \end{lstlisting}

%%%%%%%%%%%%%%%%%%%%%%%%%%%%%%%%%%%%%%%%%%%%%%%%%%%%%%%%%%%%%%%%%%%%%%%%%%%%%%%%
\section{JSON}
%%%%%%%%%%%%%%%%%%%%%%%%%%%%%%%%%%%%%%%%%%%%%%%%%%%%%%%%%%%%%%%%%%%%%%%%%%%%%%%%

\Blindtext
